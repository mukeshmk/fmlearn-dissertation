\chapter{Conclusion}

From the evaluation of the concept of Federated Meta-Learning via implemented version of the application FMLearn, there is a clear improvement in the workflow of Algorithm Selection and Hyper-Parameter Optimisation. The algorithm recommendation system was specifically designed to replicate real work scenarios, thus facilitating the algorithm selection and configuration process. Though the application FMLearn is a prototype and a proof of concept for Federated Meta-Learning, the results proved that the concept and the application saves about 86.718\% of the time when compared to the traditional process of algorithm selection in the case of small datasets and about 95.762\% of time in the case of large datasets. Apart from saving time, the results also proved the reduced use of energy, thanks to the reduced amount of time consumed, 86.718\% and 95.762\% reduction for small and large datasets respectively. This reduction in energy and time will also save money and computational resources for developers.

The biggest contribution of FMLearn in the algorithm selection and configuration process is, elimination of the repetitive and time consuming nature of the task. Other major contributions can be attributed to the accurate nature of recommendations made and the ability of the system to grow overtime. As discussed in Section \ref{knn-model} and Section \ref{model-building}, FMLearn makes better predictions when it has more data to work with. The model rebuilding/retraining techniques used in the application makes it easier for FMLearn to make better predictions and improve user experience. 