\quad\quad “Federated Meta-Learning” (FML), a concept that allows everyone to benefit from the data that is generated through software libraries including machine learning and data science libraries. We have built FMLearn, an application developed using the client-server model, which allows the exchange of meta-data about machine learning models and data in itself for the purpose of meta-learned algorithm selection and configuration. FMLearn, scikit-learn’s toy datasets along with other datasets from UCI Machine Learning repository were used and evaluated against various machine learning algorithms using GridSearchCV and Cross-Validation for which the execution time was measured. 
{In the case of scikit-learn’s breast cancer dataset, an execution time of approx. 94.24min was recorded by performing Grid Search for the best algorithm. Whereas, when FMLearn was used only 3sec was recorded to fetch and execute the best algorithm along with its model parameters. The use of FMLearn takes approx. 3sec to identify the algorithm with the best performance for this dataset.}
{Where for a large dataset like the skin segmentation, an execution time of approx. 869.74 minutes was recorded and when using FMLearn was only 3sec sec was recorded for the same.}
Overall, the use of this application allows the user to scale down the repetitive effort and time consumed in rewriting and executing code and correcting possible human errors.

%%A short summary of the problem investigated, the approach taken and the key findings. This should be around 400 words, or less.

%%This should be on a separate page.